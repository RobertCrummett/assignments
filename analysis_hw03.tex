\footline={2/20/2025\hfill Page \folio}
\def\reals{I\kern-4pt R}
\def\nats{I\kern-4pt N}
\let\oldexists\exists \def\exists{\oldexists \;}
\let\oldforall\forall \def\forall{\oldforall \,}
\def\qed{\vrule height 6pt width 6pt depth 0pt}
\parindent 0pt
\parskip 2mm


MATH~501: Homework~3\hfill Robert~Nate~Crummett
\smallskip
\hrule


\medskip{\bf Problem 1}


If $0\leq f_k\leq f$ for all $k$, then it follows that $\int_Ef_k\leq\int_Ef$.

Since the limit of $f_k$ exists, the limit infimum of the sequence also exists and is equal to the limit.

This observation motivates
$$\lim_{k\to\infty}\int_Ef_k\leq\int_Ef=\int_E\lim_{k\to\infty}f_k=\int_E\liminf_{k\to\infty}f_k$$

Since $f_k$ is a sequence of nonnegative measurable functions on $E$, Fatou's Lemma holds:
$$\int_E\liminf_{k\to\infty}f_k\leq\liminf_{k\to\infty}\int_Ef_k$$
the last two statements imply that
$$\lim_{k\to\infty}\int_Ef_k\leq\liminf_{k\to\infty}\int_Ef_k$$
But clearly the limit infimum of a sequence is always less than or equal to the sequence.
Therefore
$$\lim_{k\to\infty}\int_Ef_k=\liminf_{k\to\infty}\int_Ef_k$$
Since $\int_Ef$ is squeezed inbetween these last two terms, it also must be equal to the last two terms.
This is the requested result.\hfill\qed\kern3pt{}


\medskip{\bf Problem 2}


Let $E_1$ and $E_2$ be measurable sets contained in $E$ a measurable set
$$E_1=\{x\in E:f(x)>0\}\qquad E_2=\{x\in E:f(x)<x\}.$$
Then by the property given,
$$\int_{E_1}f=0\qquad\int_{E_2}f=0.$$
On $E_1$ and $E_2$, $f$ is strictly positive or negative respectively.
This implies that $f=0$ almost everywhere on $E_1$ and $E_2$.
Now, $E = E_1 + E_2 + Z$, where $Z=\{x\in E:f(x)=0\}$.
So $f$ is 0 almost everywhere on $E$.
Therefore, the property that $\int_Af=0$ for all measurable subsets $A$ of $E$ implies that $f$ must be zero almost everywhere on all of $E$.\hfill\qed\kern3pt{}


\medskip{\bf Problem 3}


Observe $0\leq|x^kf(x)|\leq|f(x)|$ for all $k\in\nats$.
Since $f(x)\in L(E)$, this implies $|f(x)|\in L(E)$.
But this last observation implies that $|x^kf(x)|\in L(E)$, since $|x^kf(x)|$ is bounded above by a Lebesgue measurable function.
And since $|x^kf(x)|\in L(E)$, then $x^kf(x)\in L(E)$ as well.
But this is the first thing that I was required to show.

We have just identified a Lebesgue measurable function $f(x)$ that bounds $x^kf(x)$ for all $k$.
I can invoke the general Lebesgue Dominated Convergence Theorem (LDCT) if $x^kf(x)$ monotonically decreases with $k$ to some limit point.
Clearly this limit point is $0$.
So since this is the case, I invoke the LDCT
$$\lim_{k\to\infty}\int_Ef_k(x)\buildrel\rm LDCT\over =\int_E\lim_{k\to\infty}f_k(x)=\int_E0=0$$
Which is the second requested result.\hfill\qed\kern3pt{}


\eject
{\bf Problem 4}


Let's suppose that for some measurable subset $E_0$ of $E$ that $\int_{E_0}|f|<\varepsilon$.
It is given that $|f|$ can be represented as $|f|=\lim_{k\to\infty}f_k$ for simple functions $f_k:E\mapsto[0,\infty]$.
In light of this,
$$\int_{E_0}|f|=\int_{E_0}\lim_{k\to\infty}f_k<\varepsilon.$$

Since $f$ is Lebesgue integrable so is $|f|$.
And since $f\in L(E)$, it follows that we should be able to find some finite constant $a$ such that $f<a$ for all $x$ in $E$.
These two requirements are enough to invoke the LDCT,
$$\int_{E_0}\lim_{k\to\infty}f_k\buildrel\rm LDCT\over =\lim_{k\to\infty}\int_{E_0}f_k<\varepsilon.$$
Since each $f_k$ is a simple function, taking on a finite number of values $a_1,a_2,\ldots a_i$
$$\lim_{k\to\infty}\int_{E_0}f_k=\lim_{k\to\infty}\int_{\bigcup_i E_i}f_k=\lim_{k\to\infty}\sum_i\int_{E_i}f_k=\lim_{k\to\infty}\sum_ia_{i,k}|E_i|<\varepsilon.$$
The second equality holds because of additivity of the Lebesgue measure for disjoint sets.
In order to satisfy the final inequality, $|E_i|$ must be able to be made arbitrarily small.
So this probably means that the sum of these things needs to be less than $\delta$, but I do not quite understand how to prove this final point.
\bye
