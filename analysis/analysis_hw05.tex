\input common/date.tex


\footline{\today\hfill Page~\folio}
\def\reals{I\kern-4pt R}
\def\nats{I\kern-4pt N}
\let\oldexists\exists\def\exists{\oldexists\;}
\let\oldforall\forall\def\forall{\oldforall\,}
\def\qed{\vrule height 6pt width 6pt depth 0pt}
\def\esssup{{\rm ess}\sup}
\def\almosteverywhere{\,\,\mu\hbox{\kern1pt\rm-\kern1pt a.e.}}
\parindent 0pt
\parskip 2mm


MATH~501: Homework~5\hfill Robert~Nate~Crummett
\smallskip
\hrule


\medskip{\bf Problem~1, Part~1}

Need to show that $\|f\|_\infty=\esssup|f|$ is a norm.


{\bf a)} Show $\|f\|_\infty=0\Leftrightarrow f=0$. 

$(\Leftarrow)$ Let $f=0$.
If $f=0$ then it is bounded above by $0$.
But this is the essential supremum.
Therefore $f=0$ implies that $\|f\|_\infty=0$.

$(\Rightarrow)$ Let $\|f\|_\infty=0$.
Then 
$$\|f\|_\infty=\esssup|f|=\inf\left\{\alpha:\left|\left\{x\in E:f(x)>\alpha\right\}\right|=0\almosteverywhere\right\}=0.$$
Which means that the function has non-zero measure only at or below zero.
Since $f$ cannot change sign, and $|E|>0$, the only way that $f$ can have an infinty norm equal to zero is if the function $f$ equals zero almost everywhere.
This is what was required of me to show for part a.

{\bf b)} Show $\|\lambda f\|_\infty=|\lambda|\|f\|_\infty$. 
$$\|\lambda f\|_\infty=\esssup|\lambda f|=|\lambda|\esssup|f|=|\lambda|\|f\|_\infty.$$

{\bf c)} Show $\|f+g\|_\infty\leq\|f\|_\infty+\|g\|_\infty$.

$$\|f+g\|_\infty=\esssup|f+g|\leq\esssup|f|+\esssup|g|=\|f\|_\infty+\|g\|_\infty.$$

Therefore, $\|f\|_\infty$ is a norm.\hfill\qed\kern3pt

\medskip{\bf Problem~1, Part~2}

To show $L^\infty$ is a Banach space with $\|f\|_\infty$, I need to show it is complete.

Select a Cauchy sequence $f_n$ in $L^\infty$.
Now select a subsequence of $f_n$ such that for a strictly increasing sequence of integers $n_j$ such that
$$\|f_{n_{j+1}}-f_{n_j}\|_\infty\leq{1\over2^j}.$$
Now define
$$F_m(x)=|f_{n_1}|+\sum^m_{j=1}|f_{n_{j+1}}-f_{n_j}|,$$
which is the abolute value of the telescoping series equal to $f_{n_{m+1}}$.
The limit of this sequence is $F(x)=\lim_{m\to\infty}F_m(x)$.
In the infinity norm,
$$\|F\|_\infty\leq\|f_{n_1}\|_\infty+\sum_{j=1}^\infty\|f_{n_{j+1}}-f_{n_j}\|_\infty=\|f_{n_1}\|_\infty+1.$$
This means that the absolute series $F_m(x)$ is upper bounded for every $m$.
Since the series is upper bounded and strictly increasing, it must converge in the infinity norm.
Since the absolute telescoping series $F_m(x)$ converges in the infinite norm, the non-absolute value telescoping series for $f_{n_{m+1}}$ converges abolutely to $f$ in the infinite norm.
\bye
