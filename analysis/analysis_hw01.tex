\footline={1/25/2025\hfill Page \folio}
\noindent
MATH 503: Homework 1\hfill Robert Nate Crummett
\smallskip
\hrule
\bigskip
\noindent
{\bf Problem 1}

\noindent
Given
$$\eqalign{\liminf_{k\to\infty} a_k = a &= \lim_{k\to\infty} c_k \qquad c_k = \inf_{j \geq k} a_j\cr \limsup_{k\to\infty} a_k = a &= \lim_{k\to\infty} b_k \qquad b_k = \sup_{j \geq k} a_j}$$
the sequences $b_k$ and $c_k$ are strictly decreasing and increasing respectively, and 
$$c_k \leq a_k \leq b_k \,\,\forall \,k=1,2,\ldots \eqno(1)$$
Let $\varepsilon > 0$. From equation $(1)$, it follows that there exists an $m \in I \kern -4pt N$ such that
$$\eqalign{a_k \leq b_k &< a + \varepsilon < +\infty\cr a_k \geq c_k &> a - \varepsilon > -\infty} \qquad \forall\,\, k > m\eqno (2)$$
The inequalities in $(2)$ imply
$$a -\varepsilon < a_k < a + \varepsilon \,\, \forall \,\, k > m \quad\Rightarrow\quad |a_k - a| < \varepsilon \quad\Rightarrow\quad \lim_{k\to\infty}a_k = a \eqno \hbox{\vrule height 6pt width 6pt depth 0pt}\hskip3pt$$

\noindent
{\bf Problem 2}

\noindent
Prove that for $w, z \in I\kern -4pt R^n$,
$$|w + z| \leq |w| + |z|$$

\noindent
It is clear that
$$\eqalignno{
(|w| + |z|)^2 &= |w|^2 + 2|w||z| + |z|^2&\cr
&\geq |w|^2 + 2 |w \cdot z| + |z|^2 &(3)\cr
&\geq |w + z|^2 & (4)
}$$
Line $(3)$ follows from the Cauchy-Schwarz inequality.

\noindent
The square root of both sides of equation $(4)$ completes the proof.\hfill \vrule height 6pt width 6pt depth 0pt \hskip 3pt{}
\medskip
\noindent
{\bf Problem 3}

\noindent
For $x \in I \kern -4pt R^n$ and $\delta > 0$ show that $B(x; \delta)$ is open.
\smallskip
\noindent
Let $y \in B(x; \delta)$ and $\varepsilon > 0$.

\noindent
Choose $z \in B(y; \varepsilon)$. Then if $z \in B(x; \delta) \,\, \forall \,\, z \in B(y; \varepsilon)$, then $B(y; \varepsilon) \subseteq B(x; \delta)$ and $B(x; \delta)$ is open.
$$\eqalignno{
|x - z| &= |x - y + y - z| &\cr
&\leq |x - y| + |y - z| &(5)\cr
&< \delta + \varepsilon &(6)
}$$
Equation $(5)$ follows from the triangle inequality.

\noindent
Letting $\varepsilon \to 0$ in equation $(6)$ yields $|x - z| < \delta$.
Therefore $B(y; \varepsilon) \subseteq B(x; \delta)$. \hfill \vrule height 6pt width 6pt depth 0pt \hskip 3pt{}
\medskip
\noindent
{\bf Problem 4.1}

\noindent
Assume $m \in I \kern -4pt N$. Show that the intersection of $m$ open sets $E_k$ is open.
\smallskip
\noindent
Choose $x \in \bigcap^m_{k=1}E_k$, a point in every $E_k$.

\noindent
Now construct a ball around $x$ for each $E_k$: $B(x; \delta_k) \subseteq E_k$.
Each ball with is contained in each of the sets $E_k$ because the sets are all open.
The set of the balls share the same center $x \in \bigcap^m_{k=1} E_k$.
The ball with the smallest radius will be contained in all of the other sets.
Therefore, the smallest ball is contained in the intersection of the sets, implying the intersection of $m$ open sets is open.
$$B(x; \delta) \subseteq \bigcap^m_{k=1} E_k, \qquad \delta = \min_k{\delta_k} \eqno \hbox{\vrule height 6pt width 6pt depth 0pt} \hskip 3pt$$
\eject
\noindent
{\bf Problem 4.2}

\noindent
Assume $m \in I \kern -4pt N$. Show that the union of $m$ closed sets $E_k$ is closed.
\smallskip
\noindent
If the union of $m$ closed sets $E_k$ is closed, the complement of the union will be open by definition.

\noindent
Problem 4.1 proves that the intersection of $m$ open sets is open.
De Morgan's law states
$$C \bigcup^m_{k=1} E_k = \bigcap^m_{k=1} C E_k \eqno (7)$$
$CE_k$ are all open. Therefore $(7)$ and Problem 4.1 imply $C \bigcup E_k$ is open.
The compliment of this statement will be closed by definition.
This proves the claim that the union of $m$ closed sets is closed.\hfill \vrule width 6pt height 6pt depth 0pt \hskip 3pt{}
\medskip
\noindent
{\bf Problem 5}

\noindent
${\cal C}_1$ is the collection of open subsets in $I \kern -4pt R^n$ and ${\cal C}_2$ the collection of closed subsets in $I \kern -4pt R^n$.
Show that the Borel $\sigma$-algebra $\cal B$ is the smallest $\sigma$-algebra containing ${\cal C}_2$.
\smallskip
\noindent
By definition, the open subset ${\cal C}_1 \supset E \in {\cal B}$.
Since $\cal B$ is a $\sigma$-algebra, it follows $CE \in {\cal B}$.
Since $CE \subset {\cal C}_2$ implies that ${\cal C}_2 \subset {\cal B}$.
This shows that $\cal B$ contains ${\cal C}_2$.
\smallskip
\noindent
Any $\sigma$-algebra $\Sigma \supset {\cal C}_2$ must also contain ${\cal C}_1$.
If $E \in {\cal C}_1$, then $CE = F \in {\cal C}_2$.
By definition, if $\Sigma$ contains $CE$, it also contains $E$.
Since $\cal B$ is the smallest $\sigma$-algebra containing ${\cal C}_1$, it is also the smallest $\sigma$-algebra containing ${\cal C}_2$.\hfill \vrule width 6pt height 6pt depth 0pt \hskip 3pt{}
\bye
