\input ../common/date.tex


\footline{\today\hfill Page \folio}
\def\reals{I\kern-4pt R}
\def\nats{I\kern-4pt N}
\let\oldexists\exists\def\exists{\oldexists\;}
\let\oldforall\forall\def\forall{\oldforall\,}
\def\qed{\vrule height 6pt width 6pt depth 0pt}
\parindent 0pt
\parskip 2mm


MATH~501: Homework~4\hfill Robert~Nate~Crummett
\smallskip
\hrule


\medskip{\bf Problem 1}


The three properties of a metric are 1) $d(x,x)=0$, 2) $d(x,y)=d(y,x)$, and 3) $d(x,y)\leq d(x,z)+d(z,y)$.
$d(x,y)=\int_a^b|x(t)-y(t)|dt$ satisfies these properties.
Clearly property one is satisfied.
Property two follows from
$$d(x,y)=\int_a^b|x(t)-y(t)|dt=\int_a^b|-(y(t)-x(t))|dt=\int_a^b|y(t)-x(t)|dt=d(y,x)$$
and the third property of the metric follows from the triangle inequality
$$\eqalign{d(x,y)=\int_a^b|x(t)-y(t)|dt&=\int_a^b|x(t)-z(t)+z(t)-y(t)|dt\cr&\leq\int_a^b|x(t)-z(t)|dt+\int_a^b|z(t)-y(t)|dt=d(x,z)+d(y,z)}$$
So it is clear that $d(x,y)$ is a metric.
Therefore $({\rm C}[a,b], d)$ is a metric space.\hfill\qed


\medskip{\bf Problem 2}


The three properties required to show $\|\cdot\|_p$ is a norm on $\ell^p$ are 1) $\|x\|_p=0$ if and only if $x=0$, 2) that $\|\lambda x\|_p=|\lambda|\|x\|_p$, and 3) that that triangle inequality is satisfied: $\|x+y\|_p\leq\|x\|_p+\|y\|_p$.


Because the norm is a series of non-negative elements $|\xi_i|$, a norm of 0 implies that all of the elements are zero.
And if all of the elements $\xi_i=0$, the norm on $\ell_p$ will be zero as well.
So the first property is satisfied.


The second property holds in a pretty straight forward manner as well
$$\|\lambda x\|_p=(\sum^\infty_{i=1}|\lambda\xi_i|^p)^{1/p}=(\sum^\infty_{i=1}|\lambda|^p|\xi_i|^p)^{1/p}=|\lambda|(\sum^\infty_{i=1}|\xi_i|^p)^{1/p}=|\lambda|\|x\|_p$$


The final property requires some leg work.
$$\eqalign{\|x+y\|_p^p&=\sum_{i=1}^\infty|\xi_i+\eta_i|^p\cr&=\sum_{i=1}^\infty|\xi_i+\eta_i||\xi_i+\eta_i|^{p-1}\cr&\leq\sum_{i=1}^\infty(|\xi_i|+|\eta_i|)|\xi_i+\eta_i|^{p-1}\cr}$$
where this last step follows from the triangle inequality for numbers.
Continuing,
$$\eqalign{&=\sum_{i=1}^\infty|\xi_i||\xi_i+\eta_i|^{p-1}+\sum_{i=1}^\infty|\eta_i||\xi_i+\eta_i|^{p-1}\cr&\leq(\sum_{i=1}^\infty|\xi_i|^p)^{1/p}(\sum_{i=1}^\infty|\xi_i+\eta_i|^{(p-1)({p\over p-1})})^{({p-1\over p})}+(\sum_{i=1}^\infty|\eta_i|^p)^{1/p}(\sum_{i=1}^\infty|\xi_i+\eta_i|^{(p-1)({p\over p-1})})^{({p-1\over p})}\cr&=(\|x\|_p+\|y\|_p){\|x+y\|_p^p\over\|x+y\|_p}}$$
H\"older's inequality was used to expand the sums in the second line.
Finally, the last statement is enough to prove Minkowski's inequality $\|x+y\|_p\leq\|x\|_p+\|y\|_p$.
But this is a statement of the triangle inequality for the norm $\|\cdot\|_p$, the final property of the norm.
Therefore $\|\cdot\|_p$ is a norm on $\ell^p$.\hfill\qed
\eject


{\bf Problem 3}


By definition, a Cauchy sequence is one for which for $\varepsilon>0$ there exist $m,n>N(\varepsilon)$ such that
$$d(x_m,x_n)\leq\varepsilon$$
It is given that $(x_n)_{n=1}^\infty$ is a subsequence converging to $x\in X$, or that for some $\varepsilon>0$,
$$d(x_{n_k},x)\leq\varepsilon\eqno(*)$$
Given the previous two statements, we are required to show that
$$d(x_n,x)\leq\varepsilon\eqno(1)$$
Rewriting (1) as $d(x_n,x)\leq d(x_n,x_{n_k}) + d(x_{n_k},x)$, ($*$) informs me that the second term on the right hand side can be made smaller than $\varepsilon$ with a suitable choice of $N$.
Likewise, since $n_k$ increases with $k$, we can select a $k$ such that the first term is arbirarily small, because it will behave ``Cauchily''
$$d(x_n,x_{n_k})\leq\varepsilon$$
Therefore, we choose $C=\max(n_k,N)$ and for this choice let $n\geq C$ so that $d(x_n,x)<2\varepsilon$.
Since $\varepsilon>0$ can be made as small as necessary, it is clear that (1) holds.
This is the required result.\hfill\qed


\medskip{\bf Problem 4}


Let $(x_n)_{n=1}^\infty$ be a Cauchy sequence in the normed linear space $X$.
By the {\it Cauchiness\/} of $(x_n)_{n=1}^\infty$, for $m,n\geq N$, the elements of the sequence draw close together: $\|x_n-x_m\|\leq\varepsilon$.
Now fix $n$, and let there be a sequence $(n_k)_{k=1}^\infty$ is a strictly inreasing positive integer sequence such that $n>n_k$.
Now $\|x_n-x_{n_k}\|$ can be made arbitrarily small by its Cauchiness so we make it less than $1/k^2$:
$$\|x_n-x_{n_k}\|<{1\over k^2}\eqno(2)$$


Now let $y_k=x_{n_{k+1}}-x_{n_k}$.
The sum $\sum_{k=1}^\infty y_k$ converges.
This is because the the summation converges absolutely.
This follows from taking (2) and plugging it into the abolsute sum:
$$\sum_{k=1}^\infty|y_k|=\sum_{k=1}^\infty{1\over k^2}={\pi^2\over6}\eqno(\circ)$$
Because the sum converges abosolutely, the plain jane version also converges.


Now $x_{n_k}$ can be expressed in a telescoping series
$$x_{n_k}=x_{n_1}+\sum_{j=1}^{k-1}(x_{n_{j+1}}-x_{n_j})=x_{n_1}+\sum_{j=1}^{k-1}y_k\eqno(3)$$
So in the limit as $k$ approachs $\infty$, ($\circ$) informs me that the right hand side of (3) will converge.
Therefore, $\lim_{k\to\infty}x_{n_k}$ will also converge.\hfill\qed


\medskip{\bf Problem~5}


Let $A,B\in\reals$ such that for the equivalent norms $\|\cdot\|$ and $\|\cdot\|_0$,
$$A\|\cdot\|\leq\|\cdot\|_0\leq B\|\cdot\|\eqno(\S)$$


First I will show that $(x_n)^\infty_{n=1}$ is Cauchy in $X$ with respect to $\|\cdot\|$ if the sequence is a Cauchy sequence in $X$ with respect to $\|\cdot\|_0$.
Since the elements of the sequence will be Cauchy in the norm $\|\cdot\|_0$, for $m,n>N$ and $\varepsilon>0$
$$\|x_m-x_n\|_0\leq\varepsilon$$
\eject


But by (\S) it stands to reason that
$$A\|x_m-x_n\|\leq\|x_m-x_n\|_0\leq\varepsilon$$
But this shows that the sequence is Cauchy-like in $\|\cdot\|$.
This is the first half of the proof.


The second half of the proof follows in largely the same manner.
But now, let the sequence be Cauchy in $\|\cdot\|$:
$$\|x_m-x_n\|\leq\varepsilon$$
Again by (\S),
$${1\over B}\|x_m-x_n\|_0\leq\|x_m-x_n\|\leq\varepsilon$$
Therefore Cauchiness in $\|\cdot\|$ implies the same in $\|\cdot\|_0$.
This is the second half of the proof.
Therefore, a sequence that is Cauchy in one norm will be Cauchy in any equivlanent norm on the same space.\hfill\qed
\bye
