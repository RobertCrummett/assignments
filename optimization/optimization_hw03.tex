\input ../common/date.tex


\footline{\today\hfill Page~\folio}
\def\qed{\vrule height 6pt width 6pt depth 0pt}
\parindent 0pt
\parskip 2mm


EENG~521: Homework~3\hfill Robert~Nate~Crummett
\smallskip
\hrule


\medskip{\bf Problem~1}


The sequence $x_k=1+0.5^{2^k}$ converges to 1 quadratically.
Let the error in the sequece be $e_k=1-x_k=-0.5^{2^k}$.
That this error shrinks quadraticlly can be seen by analyzing the decrease in the magnitude of the error.
$$e_{k+1}=-0.5^{2^{k+1}}=-0.5^{2^k\cdot 2}=e_k^2$$
Therefore, at each iteration $k+1$, the error is proportional to the square of the previous error $e_k$.\hfill\qed\kern3pt


\medskip{\bf Problem~2}


The limit of the sequence $x_k={1\over k!}$ is 0, and the series approaches 0 superlinearly.
As before, I will observe the error term $e_k=0-x_k=-{1\over k!}$
The next error term can be expressed as
$$e_{k+1}=-{1\over(k+1)!}=-{1\over k!(k+1)}={e_k\over(k+1)}$$
As $k$ goes to infinity,
$$\lim_{k\to\infty}{e_{k+1}\over e_k}=\lim_{k\to\infty}{1\over k+1}=0\eqno(1)$$
This shows that the rate of convergence is faster than linear, because linear convergence would be proportional to a positive constant, not zero.
However, because the limit (1) is not proportional to $e_k$, the series does not converge quadratically.
Beause the convergenve is faster than linear, but slower than quadratic, the series converges superlinearly.\hfill\qed\kern3pt


\medskip{\bf Problem~3}


The gradient and Hessian of $f_\epsilon$ are
$$\nabla f_\epsilon=\nabla f+{\epsilon\over R^2}x\qquad{\rm and}\qquad\nabla^2f_\epsilon=\nabla^2f+{\epsilon\over R^2}$$
\bye
