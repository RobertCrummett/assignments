\footline={2/20/2025\hfill Page \folio}
\def\reals{I\kern-4pt R}
\def\nats{I\kern-4pt N}
\let\oldexists\exists \def\exists{\oldexists \;}
\let\oldforall\forall \def\forall{\oldforall \,}
\def\qed{\vrule height 6pt width 6pt depth 0pt}
\parindent 0pt
\parskip 2mm


EENG~521: Homework~2\hfill Robert~Nate~Crummet\kern-1pt t
\smallskip
\hrule


\medskip{\bf Problem 1}


{\bf a)}\hskip2mm
Let $f(x)=100(x_2-x_1^2)^2+(1-x_1)^2$.
The gradient $\nabla f(x)\in\reals^2$ and the Hessian $\nabla^2f(x)\in\reals^{2\times 2}$ are
$$\eqalignno{\nabla f(x)&=\pmatrix{-400x_1(x_2-x_1^2)-2(1-x_1)\cr200(x_2-x_1^2)}&\cr\nabla^2f(x)&=\pmatrix{-400x_2+1200x_1^2+2&-400x_1\cr-400x_1&200}&\qed\hskip3pt{}}$$


To prove $x^*=(1\;1)$ is the only stationary point, it is sufficient to show that it is the unique solution to the linear system $\nabla f(x)=0$.
If a solution exists, it is unique, because there are two equations in two unknowns
$$\eqalignno{400x_1^3-400x_1x_2+2x_1-2&=0&(1)\cr-200x_1^2+200x_2&=0&(2)}$$
Equation (2) implies that $x_2=x_1^2$.
Substitution of this equality into (1) yields $x_1=1$.
Plugging this value back into (2), I find that $x_2=1$.
This yields the desired result.\hfill\qed\hskip3pt{}


To show $x^*$ is the unique local minimizer, the Hessian must be positive definite.
$$\nabla^2f(x^*)=\pmatrix{\hphantom{-}802&-400\cr-400&\hphantom{-}200}\eqno(3)$$
The eigenvalues of this matrix are $\lambda_1\approx1001.6$ and $\lambda_2\approx0.3994$.
Positive eigenvalues mean $\nabla^2f(x)\succ0$\hfill\qed\hskip3pt{}


{\bf P.S.\ }This matrix is quite poorly conditioned.
This is evident both from the large distance between the eigenvalues and the similarity in the rows of (3).


{\bf b)}\hskip2mm
Let $f(x)=8x_1+12x_2+x_1^2-2x_2^2$.
Show that there is one stationary point, and that it is a saddle point.


As before, let $\nabla f(x)=0$.
$$\nabla f(x)=\pmatrix{8+2x_1\cr12-4x_2}=0$$
Clearly, there is only one stationary point: $x^*=(-4\;3)$.
This is the first requested result.\hfill\qed\hskip3pt{}


To show $x^*$ is a saddle point, the eigenvalues of the Hessian $\nabla^2f(x^*)$ must be examined.
$$\nabla^2f(x)=\pmatrix{2&\hphantom{-}0\cr0&-4}\eqno(4)$$
The Hessian (4) has no dependence on $x$.
Therefore $\nabla^2f(x)=\nabla^2f(x^*)$.
The Hessian (4) is also already diagonal.
By inspection, the eigenvalues are $\lambda_1=2$ and $\lambda_2=-4$.
Because the eigenvalues have opposite signs, the stationary point $x^*$ must lie on a saddle.
This is the second requested result.\hfill\qed\hskip3pt{}


{\bf c)}\hskip2mm
Let $f(x)=e^{x_1+x_2}+(x_1-x_2)^2$. The gradient $\nabla f(x)$ is
$$\nabla f(x)=\pmatrix{e^{x_1+x_2}+2(x_1-x_2)\cr e^{x_1+x_2}-2(x_1-x_2)}$$
Stationary points, where $\nabla f(x)=0$, never occur.
To see this, let each component of the gradient be equal to zero.
$$\eqalign{e^{x_1+x_2}+2(x_1-x_2)=0\cr e^{x_1+x_2}-2(x_1-x_2)=0}$$
If the equations are added together, the polynomial terms cancel and the exponential terms only remain.
$$(e^{x_1+x_2}+2(x_1-x_2))+(e^{x_1+x_2}-2(x_1-x_2))=2e^{x_1+x_2}=0$$
But the exponential is positive for all points $x_1$ and $x_2$.
So the last equality has no solutions.
Therefore there are no stationary points of this function.
Therefore there is no global minimizer.\hfill\qed\kern3pt

\eject
{\bf Problem 2}


{\bf a)}\hskip2mm
To show that $f$ is convex, it is sufficient to show the Hessian is positive semi-definite.
The Hessian of $f$ is
$$\nabla^2f(x)=A^TA$$
For general $A$, $A^TA$ will be positive semi-definite.
Therefore $f$ is convex, which is the requested result.\hfill\qed\kern3pt

{\bf b)}\hskip2mm


{\bf c)}\hskip2mm
\bye
